  % interacttfssample.tex
% v1.05 - August 2017

\documentclass[dvipdfmx]{interact}

%\usepackage{epstopdf}% To incorporate .eps illustrations using PDFLaTeX, etc.
%\usepackage[caption=false]{subfig}% Support for small, `sub' figures and tables
%\usepackage[nolists,tablesfirst]{endfloat}% To `separate' figures and tables from text if required
\usepackage[format=hang]{caption}
\usepackage[format=hang, subrefformat=parens]{subcaption}
\captionsetup{compatibility=false}
\usepackage{algorithm}
\usepackage{algorithmic}
\usepackage{hyperref}
\usepackage{amsmath,amssymb}

%\usepackage[doublespacing]{setspace}% To produce a `double spaced' document if required
%\setlength\parindent{24pt}% To increase paragraph indentation when line spacing is doubled
%\setlength\bibindent{2em}% To increase hanging indent in bibliography when line spacing is doubled

\usepackage[numbers,sort&compress]{natbib}% Citation support using natbib.sty
\bibpunct[, ]{[}{]}{,}{n}{,}{,}% Citation support using natbib.sty
\renewcommand\bibfont{\fontsize{10}{12}\selectfont}% Bibliography support using natbib.sty

\theoremstyle{plain}% Theorem-like structures provided by amsthm.sty
\newtheorem{theorem}{Theorem}[section]
\newtheorem{lemma}[theorem]{Lemma}
\newtheorem{corollary}[theorem]{Corollary}
\newtheorem{proposition}[theorem]{Proposition}

\theoremstyle{definition}
\newtheorem{definition}[theorem]{Definition}
\newtheorem{example}[theorem]{Example}

\theoremstyle{remark}
\newtheorem{remark}{Remark}
\newtheorem{notation}{Notation}

\newtheoremstyle{problemstyle}  % <name>
        {3pt}                                               % <space above>
        {3pt}                                               % <space below>
        {\normalfont}                               % <body font>
        {}                                                  % <indent amount}
        {\bfseries\itshape}                 % <theorem head font>
        {\normalfont\bfseries:}         % <punctuation after theorem head>
        {.5em}                                          % <space after theorem head>
        {}                                                  % <theorem % head spec (can be left empty, meaning `normal')>
\theoremstyle{problemstyle}

\newtheorem{problem}{Problem}[section] % Comment out [section] to remove section number dependence


\begin{document}

%\articletype{ARTICLE TEMPLATE}% Specify the article type or omit as appropriate

\title{Polyhedra with Spherical Faces and ??? Four-Dimensional Kleinian Groups}

\author{
\name{Kento Nakamura, Yoshiaki Araki and \textsuperscript{a}\thanks{CONTACT
Kento Nakamura. Email: somaarcr@gmail.com Web: soma-arc.net}
 and Kazushi Ahara\textsuperscript{b}}
\affil{\textsuperscript{a}Graduate School of Advanced Mathematical
Sciences, Meiji University, Tokyo, Japan; 
\textsuperscript{b}School of Interdisciplinary Mathematical Sciences
, Meiji University, Tokyo, Japan.}
}

\maketitle

\begin{abstract}
 これまで,球面体群について,提案され
 このフラクタルは豊かな図像を見せる.
 元来,多大な時間がかかっていた.しかし,ハードウェアの進化とアルゴリズ
 ムによってリアルタイムに図を描画できるようになった.
 本論文では可視化された図像から数学的問題を提起し,解決する.
 その問題は...であり,...を示す.

 We are interested in finding examples of quasi-fuchsian subgroup of
 M\"ob$(S^3)$, the group of orientation preserving M\"obius
 transformations on $S^3$. In thiws article we consider a new problem of
 Euclidian geometry about polyhedra with spherical faces and we obtain
 interesting examples of quasi-fuchsian groups and pictures of fractal
 2-spheres.
\end{abstract}

\begin{keywords}
mathematics; visualization; fractals; Kleinian groups; circle inversion;
 sphere inversion
\end{keywords}

\section{Introduction}

We are interested in finding examples of quasi-fuchsian sub-group of
M\"ob$_{+}(S^3)$, the  group of orientation preserving M\"obius
transformation on $S^3 = R^3 \bigcup \{\infty\}$. For a subgroup $G$ of
Mob$_+(S^3)$, let the discontinuity set $\Omega(G)$ be
$$
\Omega(G) = \left\{ x \in S^3 \left| \text{the point x possesses a neighborhood} U(x)
\text{such that the intersection} U(x) \cap gU(x) \text{is empty for all but finite
elements} g\in G \right. \right\}.
$$
The complement $\Lambda(G) := S^3 \backslash \Omega(G)$ is called the
limit set of the group $G$. $G$ is a Kleinian group if $\Omega(G)$ is
not empty. The limit set $\Lambda(G)$ consists either of 0 or 1 or 2 or
infinitely many points. Here we consider only the case that $\Lambda$
is infinite and two-dimensional. A Kleinian group $G$ is a fuchsian
group if $\Lambda(G)$ is a closed round sphere. $G$ is called a
quasi-fuchsian group is $\Lambda(G)$ is homeomorphic to a closed sphere
and there exits a homeomorphism $f:S^3 \rightarrow S^3$ which conjugates
$G$ to a fuchsian group. If $G$ is a quasi-fuchsian group, the limit set
$\Lambda(G)$ may be a fractal sphere, that is, a sphere in $S^3$ with
fractal structure.

In two dimensional case, that is, $G$ is a subgroup of M\"ob$_+(S^2)$
(or equivalently M\"ob$_+(H^3)$, we have many examples of quasi-fuchsian
groups.) One of important examples is a kissing Schottky group.
Let $O_1^+, O_2^+, O_1^-, O_2^-$ be circles in $S^2$ such that
$O_1^{\pm}$ are tangential to $O_2^{\pm}$. Let $f_i$ be a M\"obius
transformation which maps the exterior of $O_i^+$ onto the interior of
$O_i^{-}$ for $i = 1, 2$ and let a group $G$ be generated by $f_1, f_2$.
(Remark that $\Lambda$ is homeomorphic to a circle if a set of the
tangent points is equvalent under the action of $G$ See [Mumford,
Series and Write 2002], p.157.) If the four circles are dusjoint to each
other, we call $G$ a Schottky group and $\Lambda(G)$ is an infinite set
but is not homeomorphic to a circle. We may consider the case that
$O_1^{\pm}$ intersects ti $O_2^{\pm}$ by $\pi/n$ for a natural number
$n$. In this case we also have a circle-wise limit set $\Lambda(G)$.
See [Mumford, Series and Write 2002], p361.

Coxeter-like group is another example. Consider 4 (or more) circles in
$S^2$ (, we call them 'initial circles'.) Let $\tilde G$ be a group generated
by the inversions of these initial circles.% Here we assume that any two
initial circles are disjoint or tangential or intersecting  by angle
$\pi/n$ for $n \in N$. Then a subgroup $G$ which is defined by a set of
all even-length-elements of $\tilde G$ is well-defined.

In this article we consideran extension of kissing Schottky groups and
Coxeter-like groups to M\"ob($A^3$). In order to do that, we have to
obtain a set of initial spheres (instead of 'initial circles') with some
conditions. In fact, the following new problem of Eucliduan geometry is
one of good approaches.

\begin{problem}
 Let $S^3 = R^3 \bigcup {\infty}$. Find a polyhedron in $S^3$ satisfying
 the following conditions.
 \begin{description}
  \item[Condition 1:] Each face of a polyhedron is a sphere in $S^3$.
  \item[Condition 2:] For each edge, the face-angle equals to $\pi/n$
             for a natural number n.
  \item[Condition 3:] For each vertex, edges with the vertex are
             mutually tangent at the vertex.
 \end{description}
\end{problem}

Usually a polyhedron has a planer faces. But easily we can consider an
extension of polyhedron such that the faces are a part of a
sphere. (Hence each edge is an arc.) If we have such polyhedron with
above conditions, we consider spheres of the faces as initial
spheres. Easily we can consider a Coxeter-like group generated by the
inversions of them.

In this article we first consider a polyhedron with the same cellular
structure as that if cybe. (Hence the number of the initial spheres is
6.) And we have the following result.

\begin{theorem}
 If a polyhedronn with the above conditions is of cube-type (that is,
 the one skeleton of the polyhedron is sane as tgat if a cube,) then
 there are 7 combinations of face-angles. For each combination of
 face-angles, a set if conformal equivalent classes of polyhedra is two
 dimensional.
\end{theorem}

Onve we have a polyhedron $P$ with the above conditions, we may consider
a Coxeter-like subgroup $G=G(P)$ of M\"ob$_+(S^3)$. Through proving this
theorem, we obtain that for each combination of face angles, there is a
polyhedron such that the limit set $\Lambda{G}$ is a round sphere. So,
other polyhedra give us quasi-fuchsian groups. So we get a family of
quasi-fuchsian group with two parameters. At the same time, we have a
family of fractal spheres with two parameters.

This paper is organized as follows. In this section2 we will show the
above theorem. We also see many more pictures of fractal spheres in the
web site [Ahara and Araki 2002]. In the section 4 we consider other type
of polyhedra and show a similar theorems.
\section{Hexahedron of cube type with spherical faces}
We will formulate hexahedron of cube type in $S^3 = R^2 \cup \{\infty\}$.
First ket $E$ be a set of edges if hexahedron of cube type. That is,
$$
E = \{(1, 2), (1, 3), (1, 4), (1, 5), (2, 3), (2, 4), (2, 6), (3, 5),
(3, 6), (4, 5), (4, 6), (5, 6)\}.
$$
Next, let $V$ be a set of vertices of a polyhedron. That is,
$$
V = \{(1, 2, 3), (1, 4, 5), (3, 5, 6), (1, 2, 4), (4, 5, 6), (2, 3, 6),
(2, 4, 6), (1, 3, 5)\}.
$$
We fix a sequence $\{n_{ij}\}_{(i,j) \in E}$ of natural numbers to
parametrize face-angles.

Under this preparation, we will consider a 6-ple($O_1$, ..., $O_6$)
satisfying the following the following conditions.

\begin{description}
 \item[(P1):] $O_i(i = 1, 2, ..., 6)$ is a sphere or a plane in $R^3
            \cup \{\infty\}$
 \item[(P2):] For each $(i, j) \in E, O_i$ and $O_j$ intersect. Let
            $e_{ik}$ be the intersection and we call it an
            \textit{edge}. And the face-angle at $e_{ij}$ is
            $\theta_{ij} = \pi/n_ij$ ($n_{ij} is a natural number.$)
 \item[(P3):] For each $(i, j, k) \in V, e_{ij}, e_{jk}$ and $e_{ik}$
            are mutually tangent at one point. Let $v_{ijk}$ be 
            the point and we call it a vertex. For simplisity of
            notation, we denote
\end{description}
$$
A = v_{123}, B=v_{145}, C = v_{356}, D = v_{124}
E = v_{456}, F=v_{236}, G = v_{246}, H = v_{135}.
$$

Let $\tilde\varepsilon(n_{ij})$ be a set if all 6-ple $(O_1, ..., O_6)$
satisfying the above conditions. We call such 6-pre $A$ polyhedron of
cube type. That is,
$$
\tilde\varepsilon(n_{ij}) = \{(O_1, ..., O_6) | (P1), (P2), (P3)\}
$$

If we parameterize $\tilde\varepsilon$ by coordinates of the center of
$O_i$, then it is naturally embedded in the Euclidean space. We consider
a natural topology on $\tilde\varepsilon$. The group M\"ob$(S3)$ of all
M\"obius transformations in $S^3$ acts $\tilde\varepsilon$ in a natural way.
Let $\varepsilon(n_{ij})$ be its quatient space. That is,

$$
\varepsilon (n_{ij}) = \tilde\varepsilon(n_{ij}) / \text{M\"ob}(S^3)
$$

We call $\varepsilon(n_{ij})$ a moduli of polyhedra.

\subsection{Preparation}

We will show sime lemmas in this subsection.

\begin{lemma}
For each $O_i$, the four vertices on $O_i$ lie on the same circle.
Hence the four vertices lie on one plane in $R^3$.
\end{lemma}

\begin{proof}
 We know the following famous lemma about four circles : If four circles 
 $C_1, C_2, C_3, C_4$ on a plane and they are mutually tangent like a
 chain as in Figure 2, then the four tangent points lie on one circle.
 (See figure 2.) Using this lemma we will show lemma 2.1. We fix $i$ and
 consider only on $O_i$. If $O_i$ is not a plane, we transform 
 ($O_1, ..., O_6$) by a M\"obius transformation such that $O_i$ is a
 plane. There are four edges on $O_i$ and they are circles and are
 mutually tangent like a chain. From the above well-known lemma. The
 four vertices on $O_i$ lie on one circle on $O_i$. If we re-transform
 ($O_i, ..., O_6$) back, the four vertices still lie on one circle,
 because any M\"obius transformation maps a planer circle to a planer circle.
\end{proof}

\begin{lemma}
 Eight vertices ofr a polyhedron of cube type lie on the same sphere.
\end{lemma}

\begin{proof}
 There is a M\"obius transformation $\phi$ such that it moves the vertex
 $H$ to the infinity point. (See Figure 3.) From Lemma 2.1, the three
 vertices $A, D,$ and $B$ lie on one straight line. In the same way,
 three vertices $A, D, G,$ and $F$ lie on one planercircle, all vertices
 except $H$ line on one plane. We transform them back by $\phi^{-1}$ and
 we get the conclusion of Lemma 2.2.
\end{proof}

\begin{lemma}
 For each vertex, there are three edges with the vertex. The sum of
 face-angles of these three edges is $\pi$.
\end{lemma}

\begin{proof}
 Let $\phi$ be a M\"obius transformation such that it maps the vertex to
 the infinity point. Then the three edges with the vertex are mapped to
 parallel lines and they form a triangle column. If we consider a
 regular section of the tiangle column then the angles of the section
 are equal to face-angles, and the sum of them is $\pi$ clearly. This
 completes the proof.
\end{proof}

 Using this Lemma 2.3, we obtain the following lemma in a combinatorical way.

\begin{lemma}
 12-ple($n_{ij}$) = ($n_{12}, n_{13}, n_{14}, n_{15}, n_{23}, n_{24},
 n_{26}, n_{35}, n_{36}, n_{45}, n_{46}, n_{56}$) of face-angle
 parameters is one of the following 11.(See Figire 4.)
\end{lemma}

\noindent(a) (3, 3, 3, 3, 3, 3, 3 ,3 ,3 ,3 ,3 ,3)\\
(b) (2, 3, 3, 3, 6, 6, 2, 3, 3, 3, 3, 3)\\
(c) (2, 3, 6, 3, 6, 3, 2, 3, 3, 2, 6, 3)\\
(d) (6, 2, 3, 3, 3, 2, 3, 6, 3, 3, 6, 2)\\
(e) (3, 2, 2, 3, 6, 6, 2, 6, 3, 6, 3, 2)\\
(f) (3, 2, 2, 3, 6, 6, 3, 6, 2, 6, 2, 3)\\
(g) (6, 2, 2, 3, 3, 3, 6, 5, 2, 6, 2, 3)\\
(h) (3, 2, 6, 3, 6, 2, 3, 6, 2, 2, 6, 3)\\
(i) (4, 2, 2, 4, 4, 4, 2, 4, 4, 4, 4, 2)\\
(j) (4, 2, 2, 4, 4, 4, 4, 4, 2, 4, 2, 4)\\
(k) (6, 2, 2, 4, 3, 3, 6, 4, 2, 4, 2, 4)\\


We get this lemma only in combinatorical way, so we omit the proof.

In this preview section we introduce that there are 7 types of
combination of face angles. But here are 11. Ub fact, in case (f), (g),
(j), and (k), there is no polyhedron with these face-angles. See section 2.4.

\subsection{Constructing hexahedron}

We fix a set($n_{ij}$) of face-angles and consider conformal equivalent
classes of polyhedra with (P1), (P2), (P3).

In the sequel we suppose that $H = \infty$. (See Figure 3) So the three
edges $HA, HB, HC$ are straight lines and are parallel to each other. We
assume that they are perpendicular to xy-plane.
Let ($x_A, y_B, z_z$), ..., ($x_G, y_G, z_G$) be coordinates of vertices
A, B, ..., G (except $H$) respectively. Let $A'(x_A, y_A)$, ...,$G'(x_G,
y_G)$ be projection of these 7 vertices to the xy-plane.

We get immediately that $\angle C'A'B' = \theta_{13}, \angle A'B'C' =
\theta_{15}, \angle B'C'A'=\theta_{13}$. Remark that $\theta_{13} +
\theta_{15} + \theta_{35} = \pi$ by Lemma2.3.

Here we may assume that the radius of the circumcircle of $A'B'C'$ is 1,
that the circumcenter of $A'B'C'$ is $(0, 0, 0)$, and that $A'$ is on
the x-axis. IF it is not, by a orioer M\"obius transformation(such that
it fixes the infinity point,) and it is easy to let the polyhedron
satisfy these conditions. By parallel transformation along z-axis, we
may assume that $A$ is on xy-plane. (That is, $A = A' and z_A = 0.$)

See Figure 3 again. $O_2, O_4, O_6$ are spheres Let $O_i(x_i, y_i, z_i)$
and $r_i$ be the center and the radius of $O_i for i = 2, 4, 6.$ About
the radius, we have the following proposition.

\begin{proposition}
 \begin{eqnarray}
 r_2sin\theta_{12} + r_4sin\theta_{14} &=& \overline{AB}^2 / (2\overline{A'B'})\\
 r_4sin\theta_{45} + r_6sin\theta_{56} &=& \overline{BC}^2 / (2\overline{B'C'})\\
 r_2sin\theta_{23} + r_6sin\theta_{36} &=& \overline{CA}^2 / (2\overline{C'A'})
 \end{eqnarray}
\end{proposition}

\begin{proof}
 In Figure 5. the edge $AD = e_{12}$ is a circle with radius
 $r_2sin\theta_{12}$, and the edge $DB e_{14}$ is a circle with radius
 ($r_4sin\theta_{14}$).
Two vertical lines $HA$ and $HB$ aare tangent to these circles as in
 Figure 5,
 and we have 
 \begin{eqnarray*}
  \overline{AD} = 2 r_2 sin\theta_{12} \cdot \overline{A'B'\|}/\overline{AB}
  \overline{DB} = 2 r_4 sin\theta_{14} \cdot \overline{A'B'\|}/\overline{AB}
 \end{eqnarray*}
 Remark that $cos(\angle BAB') =\overline{A'B'}/\overline{AB}.$ Clearly
 $\overline{AD} + \overline{DB} = \overline{AB}$ holds, so it is easy to
 show the first formula of Proposition 2.5. Others are shown in the same
 way.

\end{proof}

\begin{remark}
 If we regard ($2.A$) as linear equation on $r_2, r_4, r_6$, the
 coordinate matrix of ($2.A$) is not singular. In fact,
 \begin{eqnarray*}
  \left| 
   \begin{array}{ccc}
    sin\theta_{12} & sin\theta_{14} & 0\\
    0              & sin\theta_{45} & sin\theta_{56}\\
    sin\theta_{32} & 0              & sin\theta_{36}
   \end{array}
  \right|
 \end{eqnarray*}
 is positive.
 If we fix three vertices $A, B, C$ then always there are solutions of
 (2.A), but they may be non-positive. And, if $r_2$ or $r_4$ or $r_6$
 are too large, ($O_i$) doesn't form a hexahedron. For example, see
 section 3. In this section, we know that $r_2, r_4, r_6$ are all
 'valid' if and only if parameters ($z_B, z_C$) are in the restricted are
 area

Next, we will obtain the coordinate of the centers of $O_2, O_4, O_6$.
Let $O_i'(x_i, y_i)$bew a projection of the center of the sphere $O_i$
 to xy-plane (for i = 2, 4, 6.)
\end{remark}

\begin{proposition}
 (1)$\angle O_2'A'B' = \pi/2 - \theta_{12}, \angle
 O'_2A'C'=\pi/2-\theta_{23}$.

  $\angle O_4'B'A' = \pi/2 - \theta_{14}, \angle O_4'B'C' = \pi/2 -
 \theta_{45}$.

 $\angle O_6'C'A' = \pi/2 - \theta_{36}, \angle O_6'C'A' = \pi/2 - \theta_{56}$.

 (2)$z_A = z_2 = 0 and \overline{A'O'_2} = \overline{AO_2} = r_2$

 $z_B = z_4 and \overline{B'O'_4} = \overline{BO_4} = r_4$

 $z_C = z_6 and \overline{C'O'_6} = \overline{CO_6} = r_6$
\end{proposition}

\begin{proof}
 (1) Since the face-angle of the plane $O_1$ and the sphere$O_2$ is
 $\theta_{12}$, the angle of a straight line $A'B'$ and a circle with
 center $O'_2$ and with radius $r_2$ is also $\theta_{12}$. It follows
 that $\angle O_2'A'B' = \pi/2 - \theta_{12}$(See Figure 6.)
 
 (2) The edge AD - $e_{12}$ is the intersection of a plane $O_1$ and a
 sphere $O_2$. This edge(= plane circle) is tangent to $AH$ (which is
 parallel to z-axis) at $A$, so a radius $AO_2$ if the sphere $O_2$ is
 perpendicular to $AH$. It follows that $z_A=z_2$ and that
 $\overline{A'O'_2} = \overline{AO_2}$ immediately.

 Using Lemma 2.6(1), we can obtain all coordinates of $O_i$ from
 $\theta_{ij}$ and $r_i$ and $z_A, z_B, z_C$. In fact, the formula
 $\angle O'_2, A'B' = \pi/2 - \theta_{12}, \angle O'_{2}A'C' = \pi/2 -
 \theta_{23}$ means that the center $O_2'$ is a half-line with one
 endfloat$A'$ as in Figure 6.
 And $\overline{A'O'_2} = \overline{AO_2} = r_2$ determines $A'$.
 So $O'_2$ is uniquely determined. From Lemma 2.6(2), we get the
 z-coordinates $z_2$ of the center $O_2$ from $z_A(=0)$. Hence a sphere
 $O_2$ is uniquely determined from $\theta_{ij}$'s and positive $r_2$
 and $z_A(= 0)$.
\end{proof}

\subsection{Compatibility}
In this way for given parameters ($z_B, z_C$) such that $r_2, r_4, r_6$
are all positive, we get the coordinates of three spheres 
$O_2 O_4, O_6$. But it is not trivial whether the 6-ple($O_1, ..., O_6$) is
in $\tilde\varepsilon(n_{ij})$. (That is, we have to show that ($O_1,
..., O_6$) satisfies (P1), (P2), (P3).)

\begin{theorem}
 Fix $(n_{ij})_(i, j)\in E$(which is one of combination in Lemma 2.4)
 Fix three vertices ABC. Assume that linear equations (2.A) have
 positive solutions. Then $O_1, O_3, O_5$ as in Figure 3 and $O_2, O_4,
 O_6$ as in Proposition 2.6 satisfy followings.
 \begin{description}
  \item[(1)] The faace-angle of $O_2$ and $O_4$ (resp. $O_4$ and $O_6$,
             $O_2$ and $O_6$) is equal to $\theta_{24}$
             (resp. $\theta{46}$, $\theta_{26}$.)
  \item[(2)] The intersecrion set of two spheres $O_2$, $O_4$ and a plane
             $O_1$ consists of only one point. The point is 
             $D = v_{124}$. Similarly other two vertices $E = v_{245}$
             and $F = v_{236}$ are uniquely determined.
  \item[(3)] The intersection set of the three spheres $O_2, O_4, O_6$
             consists of only one point. The point is $G = v_{246}$.
             
  \item[(4)] $(O_1, ..., O_6) \in \tilde\varepsilon(n_{ij}))$.
\end{description}
\end{theorem}

\begin{proof}
 (1) Let $\theta'_{24}$ be the face-angle of $O_2$ and $O_4$. From the
 law pf cosine, we have
 \begin{eqnarray}
  \overline{O_2O_4}^2 = r_2^2 + r_4^2 - 2r_2r_4cos(\pi-\theta'_{24})
 \end{eqnarray}
From Figure 7, we get a formula
\begin{eqnarray*}
 \overline{O'_2O'_4}^2 = (\overline{A'B'} - r_2sin\theta_{12}-
  r_4sin\theta_{14})^2 + (-r_2cos\theta_{12} + r_4 cos \theta_{14})^2
\end{eqnarray*}
 Remark that $z_2 = z_A= 0, z_4 = z_B$. Then we have
\begin{eqnarray*}
 \overline{O_2O_4}^2 &=& \overline{O'_2O'_4}^2 + (z_2-z_4)^2\\
 &=&\overline{A'B'}^2 + 2 \overline{A'B'}(-r_2sin\theta_{12} -
  r_4sin\theta_{14}) + r^2_2 + r^2_4 -2r_2r_4cos(\theta_{12} +
  \theta_{14}) + z^2_B
\end{eqnarray*}
Using (2.B) and $\overline{A'B'}^2 + (z_A-z_B)^2 = \overline{AB}^2$,
\begin{eqnarray*}
 \overline{AB}^2 + 2\overline{A'B'}(-r_2sin\theta_{12} -
  r_4sin\theta_{14}) - 2r_2r_4\{\cos(\theta_{12} + \theta_{14}) -
  \cos(\pi- \theta'_{24})\} = 0
\end{eqnarray*}
 $r_2, r_4$ satisfy (2.A) and they are positive, so we obtain
\begin{eqnarray*}
 \cos(\theta_{12} + \theta_{14}) = \cos(\pi - \theta'_{24})
\end{eqnarray*}
From Lemma 2.3, $\theta_{12} + \theta_{14} + \theta_{24} = \pi$. It
 follows that $\theta_24 = \theta'_24$. In the same way $\theta_{46}$
 (resp.$\theta{26}$)is equal to the face-angle of $O_4$and $O_6$(resp
 $O_2$ and $O_6$.)

(2) First we show that two edges $e_{12}$ and $e_{14}$ are mutually tangent
 on $O_1$. The radius of $e_{12}$(resp. $e_{14}$) is
 $r_2\sin\theta_{12}$ (resp. $r_4\sin\theta_{14}$). Let $d$ be the
 distance between two centers of two circles $e_{12}$, $e_{14}$. See
 Figure 8.

We have
 \begin{align*}
d^2&= (\overline{A'B'} - r_2\sin\theta_{12}-r_4\sin\theta_{14})^2 + \overline{BB'}^2\\
&= (\overline{A'B'} - \overline{AB}/(2\overline{A'B'}))^2 + \overline{BB'}^2\\
&= \overline{A'B'}^2 - \overline{AB}^2 +\overline{AB}^4/(4\overline{A'B'}^2) + \overline{BB'}^2\\
&=(r_2\sin\theta_{12} + r_4\sin\theta_{14})^2\\
d&= r_2\sin\theta_12 + r_4\sin\theta_{14}
 \end{align*}
It means that $e_{12}$ and $e_{14}$ are mutually tangent.

Next we will show that the edge $e_{24}$ is tangent to both of $e_{12}$ and
 $e_{14}$. Assume that it is not. Then $e_{24}$ intersects to $O_1$
 twice.
(Remark that the tangent point of $e_{12}$ and $e_{14}$ is also on the
 circle $e_{24}$.) But all intersection points of $e_{24}$ and $O_1$ are
 on $e_{12}$ and $e_{14}$ by definition. This contradicts to the fact
 that $e_{12}$ is tangent to $e_{14}$. So the edge $e_{24}$ is tangent
 to both of $e_{12}$ and $e_{14}$. Similarly we will show it about $E$
 and $F$.

(3) Let $\alpha$ be a plane containing 3 centers $O_2, O_4, O_6$. Let
 $C_2, C_4, C_6$ be three circles on $\alpha$ such that $C_i$ is the
 intersection of the sphere $O_i$ and the plane $\alpha$. ($i=2, 4, 6$.)
Let R be the radical point of $C_2, C_4, C_6$. There are three cases
 here.
(i) $R$ us inside if $C_2, C_4, C_6$.(ii) R is outside $C_2, C_4, C_6$.
(iii)$R$ is on $C_2, C_4, C_6$. (See Figure 9)

In the case (i), let three points $S_{24}, S_{46}, S_{26}$ be as in Figure
 9 (left). Here we have $\angle O_2 S_{24} O_4 = \pi - \theta_{24}$,
because their face-angle of $O_2$ and $O_4$is $\theta_{24}$.
In the same way, $\angle O_4\theta_{46}O_6 = \pi - \theta_{46}$ and
$\angle O_2S_{26}O_6 = \pi-\theta_{26}$ hold. Hence

\begin{align*}
 \angle O_2 S_{24} O_4 + \angle O_4 S_{46} O_6 + \angle O_2 S_{26} O_6 = 3\pi -
 \theta_{24} - \theta_{46} - \theta_{26} = 2\pi.
\end{align*}

On the other hand, from Figure 9 (left)

\begin{align*}
 \angle O_2 S_{24} O_4 &< \angle O_2 R O_4
 \angle O_2 S_{24} O_4 &< \angle O_2 R O_4
 \angle O_2 S_{24} O_4 &< \angle O_2 R O_4
\end{align*}
and
\begin{align*}
 \angle O_2 R O_4 + \angle O_4 R O_6 + \angle O_2 R O_6 = 2\pi
\end{align*}
This is a contradiction. So this case cannnot happen. In the same way,
 the case (ii) cannnot happen. These follow that the radical point $R$
 is on the intersection of 3 circles $C_2, C_4, C_6.$ this means that
 $R$ isthe unique intersection of 3 spheres $O_2, O_4, O_6$. this must
 $G$.

(4) The 6-pre($O_1, ..., O_6$ satisfies (P1) by definition.

(P2): For (i, j) = (1, 3), (3, 5), (1, 5), it is good because we first
 take a triangle prism (consisting of $O_1, O_3, O_5$) with face-angles
 $\theta_{13}, \theta_{35}, \theta_{15}.$ For (i, j) = (1, 2)
, (1, 4), (4, 5), (5, 6), (2, 3), (3, 6), it is good from Lemma2.6. In
other cases, it is goodfrom Theorem 2.7(1).

(P3): At $H = \infty, AH, BH, CH$ are parallel lines so they are tangent
 at $\infty$. At $D, R, F, G,$ we have shown it in Theorem 2.7(2)(3)
\end{proof}

\subsection{Parameter space}
In this subsection we first consider in the case that $n_{ij} = 3$ for all
$(i, j) \in E$. (That is, in case (a) in Lemma 2.4.) There are 11 cases
in Lemma 2.4. The case (a) is not a special case, amd we can make
similar observation in other cases.
The linear equations (2.A) are:
\begin{align*}
 r_2 + r_4 &= (3 + z_B^2) / 3 \\
 r_4 + r_6 &= (3 + (z_B^2 - z_C)^2 ) / 3 \\
 r_2 + r_6 &= (3 + z_C^2) / 3
\end{align*}

Here $\overline{A'B'} = \overline{B'C'} = \overline{A'C'} = \sqrt{3}$
because $\Delta{A'B'C'}$ is a regular triangle and the radius of the
circumcircle is 1.(In all cases, $\overline{A'B'} = 2\sin\theta_{35},
\overline{B'C'} = 2\sin\theta{13}, \overline{A'C'} = 2\sin\theta{15}$.)
We solve these equations and we have 
\begin{align}
 r2 &= (2-z_Bz_C + 2Z^2_C + 3) / 6 \\
 r4 &= (2z_Bz_C + 3) / 6 \\
 r6 &= (2z^2_B - 2z_Bz_c + 3) / 6.
\end{align}
First we assume that $r_2, r_4, r_6$ are positive. This condition is
equivalent to the followings.
\begin{align}
 -3 / 2 &< -z_bz_C + z^2_C\\
 -3 / 2 &< z_Bz_C \\
 -3 / 2 &< z^2_B - z_Bz_C
\end{align}
So we assume (3.B). We will determine $O_1, ... , O_6$ as in the
previous section. It is clear that 6 faces $O_1, ... , O_6$ surround
a hexahedron if and only if $O_2$ doesn't interest to $O_5$ nor $O_3$
doesn't intersect to $O_4$ nor $O_1$ doesn't intersect to $O_6$.

$O_2$ doesn't interest to $O_5$ if only if the distance from the center
of the sphere $O_2$ to the plane $O_5$ is more than $r_2$. In this case
the distance is $3 / 2 - r_2$ and hence $O_2$ doesn't intersect to $O_5$
if and only if $_2 < 3/4$ and $r_6 < 3 / 4$ These three inequalities are
equivalent to
\begin{align}
 -z_Bz_C + z^2_C &< 3 / 4\\
 z_Bz_C &< 3 / 4 \\
 z^2_B - z_B z_C &< 3/ 4
\end{align}

See Figure 10 ($z_B, z_C$) in the shaded area of this figure  satisfies
all of(2.D) an (2.E) and $z_B \geq 0$.

In case (a), original three parameters $z_A, z_B, z_C$ have symmetry,
because $O_1, O_3, O_5$ from a \textit{regular} triangle prism.
If $z_C < 0$ then we exchange the roles of $z_A$ and $z_C$ and we can reduce
($z_B, z_C$) to ($z_B - z_C, -z_C$) in the case of $z_C > 0$.
If $z_B \leq z_C < z_B$ then we exchange the roles of $z_B$ and $z_C$
and we can reduce $(z_B, z_C)$ to $(z_C -z_B,z_C)$ in case of 
$z_C \geq 2z_B$. In this way, the correct parameter space is as in Figure
11.

Here two hyperbola $z^2_C - z_bz_C = 3/4$ and $z_B z_C = 3/4$ intersects
at ($\sqrt{6} / 4, \sqrt{6}/2$) and this point is on the straight line
$z_C=2z_B$. In Figure 11, points on $z^2)C 0 z_Bz_C = 3/4$ aren't
contained in the area. When ($z_b, z_C$) is on this parabola, the face
$O_2$ is tangent to $O_5$. Moreover when
$(z_B, z_C) = (\sqrt{6} / 4, \sqrt{6}/2)$, the face $O_3$ is alsotamgemt
to $O_4$.
In the same way, we have a similar resi;t om case $(b), (c), (d), (e),
(h), (i)$. The figures below are parameter space in each case.

In case $(f), (g), (j) and (k)$, there are no solution of $(z_B, z_C)$
to be a hexahedron. In fact, in case (f),
\begin{align*}
\begin{cases}
 0 &< \dfrac{z_Bz_C}{\sqrt{3}} < \dfrac{\sqrt{3}}{4} \\
 0 &< \dfrac{1 + z^2_B - z_Bz_C}{2} < \dfrac{1}{2} \\
 0 &< \dfrac{3 + z^2_C - z_Bz_C}{2\sqrt{3}} < \dfrac{\sqrt{3}}{2}.
\end{cases}
\end{align*}
In case (g),

\begin{align*}
\begin{cases}
0 &< \dfrac{z_Bz_C}{2} < \dfrac{\sqrt{3}}{4} \\
0 &< \dfrac{2 + 2z^2_B - z_Bz_C}{4} < \dfrac{1}{2} \\
0 &< \dfrac{\sqrt{3}(6 + 2z^2_C - 3z_Bz_C)}{12} < \dfrac{\sqrt{3}}{2}
\end{cases}
\end{align*}

In case (j),

\begin{align*}
 \begin{cases}
  0 &< \dfrac{z_Bz_C}{2} < \dfrac{1}{2}\\
  0 &< \dfrac{\sqrt{2}(2 + z^2_B) - z_B z_C}{4} < \dfrac{\sqrt{2}}{2}\\
  0 &< \dfrac{\sqrt{2}(2 + z^2_C) - z_B z_C}{4} < \dfrac{\sqrt{2}}{2}
 \end{cases}
\end{align*}

In case (k),

\begin{align*}
 \begin{cases}
  0 &< \dfrac{2z_Bz_C}{\sqrt{6} + \sqrt{2}} < \dfrac{1}{\cos(\pi/12+1)}\\
  0 &< \dfrac{\sqrt{2}(2 (1 + \sqrt{3}) + (1 + \sqrt{3})z^2_B -2z_Bz_C)}
  {2(\sqrt{6} + \sqrt{2})} < \dfrac{\sqrt{2}}{2}\\
  0 &< \dfrac{\sqrt{2}(2 (1 + \sqrt{3}) + (1 + \sqrt{3})z^2_C
  -2\sqrt{3}z_Bz_C)}
  {2(\sqrt{6} + \sqrt{2})} < \dfrac{\sqrt{2}}{2}
 \end{cases}
\end{align*}
In these cases, there is no answer for ($z_b, z_C$).

We complete the proof of the main Theorem 1.1.

\subsection{Group $G$ and the limit set $\Delta(G)$}
Let $\gamma_i \in \text{M\"ob}(S^3)$ be the inversion of $O_i$ for $i =
1,...,6$.
Ket $G'$ be a group generated by $\gamma_i$'s. From Condition (P2), we
have relations
\begin{align*}
 (\gamma_i \gamma_j)^{n_{ij}} = 1
\end{align*}
for $(i, j) \in E.$ (Here 1 is the identity map.) If $(i, j) \notin E$,
there are no relation between $\gamma_i$ and $\gamma_j$.
And at each vertex, three edges with the vertex are mutually tangent,
there are no more relations. So we have that
\begin{align*}
 \tilde G = <\gamma_1, ...,\gamma_6 | (\gamma_i\gamma_j)^{n_ij} = 1 for
 (i, j) \in E.>
\end{align*}
The length of all relations are even, so we can consider a set G = G(P)
of elements of even length. $G$ is a subset of $G''$. Here we remark
that each $\gamma_i$ is a inversion and it is orientation reversing. So
each element of $G$ is orientation preserving M\"obius transformation.
It follows that $G$ is a subset of M\"ob$_+(S^3)$.

In the previous subsection we got a parameter space for $P$ to be a
hexahedron with the initial conditions. Here are some pictures of the
limit sets $\Gamma(G)$. However, if we take $O_1, ..., O_6$ as in the
previous section, the limit set contains a infinity point $\infty$. So
we transform $\O_1, ..., O_6$ by an inversion of a proper sphere in
order to make $\Gamma(G)$ doesn;t contain the infinity point.

These are pictures of three-dimensional set, so it is hard to see whole
appearance only by them. We can see more pixtures and more mocies in the
web-site [Ahara and Araki 2002].

Only when $(z_B, z_C) =(0, 0), \Delta(G)$ is a round sphere and hence
$G$ is fuchsian. In order cases we get a homotopy to the case 
$(z_B, z_C) = (0, 0)$, we know that $G$ is quasi-fuchsian.

\section{Extended Schottky group for a cube with spherical faces}
If $(O_1, ..., O_6)$ gives a polyhedron with conditions$(P1), (P2),
(P3)$.
These spheres also define a Schottky type subgroup of M\"ob$_+(S^3)$ on
ly in case (a). Indeed we have the following  proposition.

\begin{proposition}
Let $\{n_{ij}\}$ be of type (a) in Lemma 2.4. Then there exist three
 (orientation preserving) transformations $f)1, f_2, f_3$ such that
 \begin{description}
  \item[(1)] $f_1$ maps the inside of $O_1$ onto the outside of
             $O_6$, $f_1(H) = C$, $f_1(A) = F$,  $f_1(D) = G$, and
             $f_1(B) = E$.
  \item[(2)] $f_2$ maps the inside of $O_3$ onto the outside of
             $O_4$, $f_2(H) = B$, $f_2(A) = D$,  $f_2(F) = G$, and
             $f_1(C) = E$.             
  \item[(3)] $f_3$ maps the inside of $O_5$ onto the outside of
             $O_2$, $f_3(H) = A$, $f_3(B) = D$,  $f_3(E) = G$, and
             $f_3(C) = F$.
\end{description}
\end{proposition}

Let $H$ be a sub group of M\"ob$_+(S^3)$ generated by $f_1, f_2,
f_3$. This proposition follows that
\begin{align*}
 H = <f_1, f_2, f_3|(f_1f_2)^3 = (f_2f_3)^3 = (f_1f_3)^3 = e>.
\end{align*}
It is easy to show that the fundamental domain of $H$ is a polyhedron
surrounded by the initial spheres. The limit set $H$ is the same as that
of $G$.

\begin{proof}
See Figure 3. Since we assume that $n_{ij} = \pi/3$, we have
\begin{align*}
 A(1, 0, 0), B(-\frac{1}{2}, z_B), C(-\frac{1}{2}, -\frac{\sqrt{3}}{2}, z_C).
\end{align*}
\end{proof}

Let $O_2$ be the center of the spheres $O_2$, then we have
\begin{align*}
 O_2(1 - r_2, 0, 0).
\end{align*}
From Lemmma2.2, 7points A, B, C, D, E, F, G lie on one plane. D is on
the segment $AB$ and
\begin{align*}
 AD : DB = r_2\sin\theta_{12} : r_4\sin\theta_{14} = r_2 : r_4
\end{align*}
See Figure 5. In the same way we have
\begin{align*}
 BE : EC = r_4\sin\theta_{45} : r_6\sin\theta_{56} = r_4:r_6
 CF : FA = r_6\sin\theta_{36} : r_2\sin\theta_{23} = r_6:r_2
\end{align*}
Using these, we have coordinates of $E$. In fact,
\begin{align*}
 E(-\dfrac{1}{2},\dfrac{\sqrt{3}}{2}\dfrac{r_6-r_4}{r_6 + r_4},
 \dfrac{z_Br_6 + z_C r_4}{r_6 + r_4})
\end{align*}
And we have
\begin{align*}
 \dfrac{AD}{DB} \cdot \dfrac{BE}{EC} \cdot \dfrac{CF}{FA} = \dfrac{r_2r_4r_6}{r_4r_6r_2} = 1
\end{align*}
From Ceva's theorem, the three lines $AE$, $BF$, and $CD$ meet at one
point. Let $G'$ be the intersection. Using Menelaus's theorem, we have
$AG':G'E = r_2r_4 + r_2r_6:r_4r_6$ and
\begin{align*}
 G' = \dfrac{1}{R}(r_4r_6 - \dfrac{1}{2}(r_2r_4 + r_2r_6), 
 \dfrac{\sqrt{3}}{2}r_2(r_6 - r_4), r_2(z_Br_6 + z_Cr_4)),
\end{align*}
where $R = r_2r_4 + r_4r_6 + r_6r_2.$ It follows that
\begin{align*}
 \overline{O_2G'}^2 = (1 - r_2 - \dfrac{1}{2R}(2r_4r_6 - (r_2r_4 +
 r_2r_6)))^2 + \dfrac{3r^2_2}{4R^2}(r_6 - r_4)^2 +
 \dfrac{r_2^2}{R^2}(z_Br_6 + z_Cr_4)^2
\end{align*}
Using Proposition 2.5 we have $\overline{O_2G'} = r_2$. Remark that we
use equations $z_B^2 = 3(r_2 + r_4 - 1), z_C^2 = 3(r_2 + r_6 - 1)$, and
$2z_Bz_C = 6r_2 - 3$.

In the same way we get $\overline{O_4G'} = r_4$ and 
$\overline{O_6G'} = r_6$. Thesemean that $G' = G$.

From Lemma 2.1 $B, D, G, E$ are on one circle. Hence
$CE \cdot CB = CG \cdot CD$. In the same way we obtain 
$CG \cdot CD = CF \cdot CA$. So let $O$ be a sphere with center $C$ and
radius $\sqrt{CE \cdot CB}$ and let $f_O$ be the inversion of $O$
The above two equations follows that
\begin{align*}
 f_O: A \mapsto F, D \mapsto G, B \mapsto E, H \mapsto C.
\end{align*}
Let $f_{O_1}$ be the inversion of $O_1$. $f_{O_1}$ fixes $H, A, D, B$.
Let $f_1 := f_O \circ f_{O_1}$.
$f_1$ is an orientation preserving M\"obius transformation and
\begin{align*}
 f_1 : A \mapsto F, D \mapsto G, B \mapsto E, H \mapsto C.
\end{align*}
It is easy to check that $f_1(O_1) = O_6$ because a sphere is determined
by four points on it. And also $f_1$ maps the inside of $O_1$ to the
outside of $O_6$. Here the inside of $O_1$ means the area which doesn't
contain the prism consistint of $O_1, O_3, O_5$. The we finish the
proof.

Remark:
In other cases (b),...,(j), we don't have a similar result. Three
segments $AR, BF, CD$ meets at one point only in case (a), (f), (j), but
there isn't a parameter space in cases (f) nor (j).

\section{Other polyhedra}
We may consioder the same problem for other polyhedra. Here is a list of
pentahedra and hexahedra with the number of combination of face-angles
and the dimension of $\tilde{\varepsilon}(n_{ij})$.

\section{Visualization}
We use ray tracing technique.
Ray marching
See also \cite{bridges2018}
3D print
\section{Problems}
\section{Summary \& Future Work}

Kazushi Ahara and Yoshiaki Araki invented a new geometrical concept
called a \textit{sphairahedron}.

Visualization of a quasi-sphere takes long time.
We developed an algorithm to render this kind of fractals fast
\cite{bridges2018}.
It is called \textit{Iterated Inversion System (IIS)}.

In this paper, we briefly introduce sphairahedron and its rendering
technique.
Next, we show problems found by viewing masthematical problems.
We find mathematical problems and solve the problems.


章立て
Introduction 球面体群について.これまでの研究成果について阿原・荒木
(classification problem),蔭山の成果
Preparation 球面体の定義など 離散的,非離散的について 理想的かつ有理的
Visualization 可視化方法について
Problems (Propositions) 図を見ることで考察することのできる数学的問題を提起,解決
Summary \& Future work 解決したこと,未解決問題のまとめ

単連結な部分が0のときはどうする? 最初から抜いておく?

\cite{bridges2018}では,球面体の定義とその図像の豊かさについて論じた.


Kazushi Ahara prooved that the limit set of the sphairahedra
is a union of the spheres.

球面体の定義
二次元における円辺形によるフラクタル(クライン群)について

(阿原・荒木がやったこと)
Yoshiaki Araki and Kazushi Ahara found Sphairahedron in 2003.
It is extension of two dimensional tessellation of circles.
円辺形を拡張した図形である球面体を用いる.
cubeタイプのパラメータに関して計算を行ない,パラメータスペースを求めた.
その求め方の論文は出版されていない.
球面体そのものでなく,球面体を含む球の和集合に群を作用させることによっても
quasi-sphereを得ることができる.その証明は...
立方体をベースにした球面体群の分類問題を扱った.
今回はレンダリングされた図から得られた数学的問題を解決する

Kazushi Ahara prooved that the limit set of the sphairahedra
is a union of the spheres.

(蔭山さんがやったこと)
In 2016 Ryo Kageyama had supplementary examination in his master thesis.
He calculate parameter space of the cube type sphairahedron.
三次元でレンズ空間を考えた.
阿原・荒木と別にcubeタイプのパラメータスペースを求めた
日本語論文である

(中村がやったこと)
球面体群の可視化は長らく時間のかかる処理であった.
IISによってリアルタイムに可視化することができるようになった.
Visualize sphairahedron(semi-sphairahedron)-based fractals in real time.
The limit set is many spheres touching each other and,
and it becomes not Fuchsian group,
準球面体の可視化にも成功した.
semi-sphairahedronの極限集合は球が多数接した形状になっており,フックス群
でなくなる.
The limit set of the semi-sphairahedron
Kento Nakamura compute parameter spaces for other sphairahedra.
別の立体におけるパラメータ表示を計算し,新たな形状を見ることができた

可視化によって見えること
Ahara Arakiによって計算されたパラメータスペースの外側にも
離散的になるパラメータが混在してい可能性がある.
それは点か,曲線か,領域か,わかっていない
There may be parameters which is discrete.
We don't know it is point, curved line, or regions.

ビジュアライズされて図から読みとれること
固定点付近は収束が遅いこと
橋ができたときにPI/nになった,
球面体の各頂点の収束が特に遅い(タイリングするためにはたくさんの球面体が必要)

七角形以上のSphairahedronはパターンが多くなること,条件を満たす
パラメータが厳しくなることから,計算することが難しい
角度の条件から
五つ以上の辺が繋がる頂点があってはいけない


(まだできていないこと)
Find mathematical property of the fractals.
実験が必要?
地形タイプの場合どこまで高くなるか(低くなるか)
橋がかかった場合の離散条件(pi/n)
The limit set of the semi-sphairahedron is not known.
However, we visualize semi-sphairahedron and its tessellation.

図を観察して こういうことができるという主張
パラメータ空間の形状はフラクタル形状がない
パラメータスペースが連結

ゴーストサークルの概念が三次元にもある
ゴーストスフィア


\begin{thebibliography}{99}
\bibitem{AharaAraki}
        Kazushi Ahara and Yoshiaki Araki,
        \emph{Sphairahedral approach to parameterize visible three
        dimensional quasi-Fuchsian fractals},
        Proceedings of Computer Graphics International, 2003, pp. 226--229.
\bibitem{bridges2018}
        Kento Nakamura and Kazushi Ahara,
        \emph{Sphairahedra and Three-Dimensional Fractals}, 
        Proceedings of Bridges 2018: Mathematics, Music, Art, Architecture,
        Education, Culture, Tessellations Publishing,
        Phoenix, Arizona, 2017, pp. 171--178.
\end{thebibliography}
\end{document}
