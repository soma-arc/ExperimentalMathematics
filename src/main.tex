% interacttfssample.tex
% v1.05 - August 2017

\documentclass[dvipdfmx]{interact}

%\usepackage{epstopdf}% To incorporate .eps illustrations using PDFLaTeX, etc.
%\usepackage[caption=false]{subfig}% Support for small, `sub' figures and tables
%\usepackage[nolists,tablesfirst]{endfloat}% To `separate' figures and tables from text if required
\usepackage[format=hang]{caption}
\usepackage[format=hang, subrefformat=parens]{subcaption}
\captionsetup{compatibility=false}
\usepackage{algorithm}
\usepackage{algorithmic}
\usepackage{hyperref}

%\usepackage[doublespacing]{setspace}% To produce a `double spaced' document if required
%\setlength\parindent{24pt}% To increase paragraph indentation when line spacing is doubled
%\setlength\bibindent{2em}% To increase hanging indent in bibliography when line spacing is doubled

\usepackage[numbers,sort&compress]{natbib}% Citation support using natbib.sty
\bibpunct[, ]{[}{]}{,}{n}{,}{,}% Citation support using natbib.sty
\renewcommand\bibfont{\fontsize{10}{12}\selectfont}% Bibliography support using natbib.sty

\theoremstyle{plain}% Theorem-like structures provided by amsthm.sty
\newtheorem{theorem}{Theorem}[section]
\newtheorem{lemma}[theorem]{Lemma}
\newtheorem{corollary}[theorem]{Corollary}
\newtheorem{proposition}[theorem]{Proposition}

\theoremstyle{definition}
\newtheorem{definition}[theorem]{Definition}
\newtheorem{example}[theorem]{Example}

\theoremstyle{remark}
\newtheorem{remark}{Remark}
\newtheorem{notation}{Notation}

\begin{document}

%\articletype{ARTICLE TEMPLATE}% Specify the article type or omit as appropriate

\title{Sphairahedron}

\author{
\name{Kento Nakamura\textsuperscript{a}\thanks{CONTACT
Kento Nakamura. Email: somaarcr@gmail.com Web: soma-arc.net}
 and Kazushi Ahara\textsuperscript{b}}
\affil{\textsuperscript{a}Graduate School of Advanced Mathematical
Sciences, Meiji University, Tokyo, Japan; 
\textsuperscript{b}School of Interdisciplinary Mathematical Sciences
, Meiji University, Tokyo, Japan.}
}

\maketitle

\begin{abstract}
 This is abstract.
\end{abstract}

\begin{keywords}
mathematics; visualization; fractals; Kleinian groups; circle inversion;
 sphere inversion
\end{keywords}

\section{Introduction}

In this paper we will

See also \cite{bridges2018}

Kazushi Ahara prooved that the limit set of the sphairahedra
is a union of the spheres.

(阿原・荒木がやったこと)
Yoshiaki Araki and Kazushi Ahara found Sphairahedron in 2003.
It is extension of two dimensional tessellation of circles.
cubeタイプのパラメータに関して計算を行ない,パラメータスペースを求めた.
その求め方の論文は出版されていない.

Kazushi Ahara prooved that the limit set of the sphairahedra
is a union of the spheres.

(陰山さんがやったこと)
In 201x Ryo Kageyama had supplementary examination.
He calculate parameter space of the cube trypo sphairahedron.
三次元でレンズ空間を考えた.
阿原・荒木と別にcubeタイプのパラメータスペースを求めた

(中村がやったこと)
Visualize sphairahedron(semi-sphairahedron)-based fractals in real time.
The limit set is many spheres touching each other and,
and it becomes not Fuchsian group,
semi-sphairahedronの極限集合は球が多数接した形状になっており,フックス群
でなくなる
The limit set of the semi-sphairahedron
Kento Nakamura compute parameter spaces for other sphairahedra.
別の立体におけるパラメータ表示を計算した.

可視化によって見えること
Ahara Arakiによって計算されたパラメータスペースの外側にも
離散的になるパラメータが混在してい可能性がある.
それは点か,曲線か,領域か
There may be parameters which is discrete.
We don't know it is point, curved line, or regions.

ビジュアライズされて図から読みとれること
固定点付近は収束が遅いこと 橋ができたときにPI/nになった,
各頂点の収束が特に遅い

七角形以上のSphairahedronはパターンが多くなること,条件を満たす
パラメータが厳しくなることから,計算することが難しい
角度の条件


(まだできていないこと)
Find mathematical property of the fractals.
実験が必要
地形タイプの場合どこまで高くなるか(低くなるか)

The limit set of the semi-sphairahedron is not known.
However, we visualize semi-sphairahedron and its tessellation.



\begin{thebibliography}{99}
\bibitem{AharaAraki}
        Kazushi Ahara and Yoshiaki Araki,
        \emph{Sphairahedral approach to parameterize visible three
        dimensional quasi-Fuchsian fractals},
        Proceedings of Computer Graphics International, 2003, pp. 226--229.
\bibitem{bridges2018}
        Kento Nakamura and Kazushi Ahara,
        \emph{Sphairahedra and Three-Dimensional Fractals}, 
        Proceedings of Bridges 2018: Mathematics, Music, Art, Architecture,
        Education, Culture, Tessellations Publishing,
        Phoenix, Arizona, 2017, pp. 171--178.
\end{thebibliography}
\end{document}
