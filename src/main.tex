  % interacttfssample.tex
% v1.05 - August 2017

\documentclass[dvipdfmx]{interact}

%\usepackage{epstopdf}% To incorporate .eps illustrations using PDFLaTeX, etc.
%\usepackage[caption=false]{subfig}% Support for small, `sub' figures and tables
%\usepackage[nolists,tablesfirst]{endfloat}% To `separate' figures and tables from text if required
\usepackage[format=hang]{caption}
\usepackage[format=hang, subrefformat=parens]{subcaption}
\captionsetup{compatibility=false}
\usepackage{algorithm}
\usepackage{algorithmic}
\usepackage{hyperref}

%\usepackage[doublespacing]{setspace}% To produce a `double spaced' document if required
%\setlength\parindent{24pt}% To increase paragraph indentation when line spacing is doubled
%\setlength\bibindent{2em}% To increase hanging indent in bibliography when line spacing is doubled

\usepackage[numbers,sort&compress]{natbib}% Citation support using natbib.sty
\bibpunct[, ]{[}{]}{,}{n}{,}{,}% Citation support using natbib.sty
\renewcommand\bibfont{\fontsize{10}{12}\selectfont}% Bibliography support using natbib.sty

\theoremstyle{plain}% Theorem-like structures provided by amsthm.sty
\newtheorem{theorem}{Theorem}[section]
\newtheorem{lemma}[theorem]{Lemma}
\newtheorem{corollary}[theorem]{Corollary}
\newtheorem{proposition}[theorem]{Proposition}

\theoremstyle{definition}
\newtheorem{definition}[theorem]{Definition}
\newtheorem{example}[theorem]{Example}

\theoremstyle{remark}
\newtheorem{remark}{Remark}
\newtheorem{notation}{Notation}

\begin{document}

%\articletype{ARTICLE TEMPLATE}% Specify the article type or omit as appropriate

\title{Polyhedra with Spherical Faces and ??? Four-Dimensional Kleinian Groups}

\author{
\name{Kento Nakamura, Yoshiaki Araki and \textsuperscript{a}\thanks{CONTACT
Kento Nakamura. Email: somaarcr@gmail.com Web: soma-arc.net}
 and Kazushi Ahara\textsuperscript{b}}
\affil{\textsuperscript{a}Graduate School of Advanced Mathematical
Sciences, Meiji University, Tokyo, Japan; 
\textsuperscript{b}School of Interdisciplinary Mathematical Sciences
, Meiji University, Tokyo, Japan.}
}

\maketitle

\begin{abstract}
 これまで,球面体群について,提案され
 このフラクタルは豊かな図像を見せる.
 元来,多大な時間がかかっていた.しかし,ハードウェアの進化とアルゴリズ
 ムによってリアルタイムに図を描画できるようになった.
 本論文では可視化された図像から数学的問題を提起し,解決する.
 その問題は...であり,...を示す.
\end{abstract}

\begin{keywords}
mathematics; visualization; fractals; Kleinian groups; circle inversion;
 sphere inversion
\end{keywords}

\section{Introduction}
Kazushi Ahara and Yoshiaki Araki invented a new geometrical concept
called a \textit{sphairahedron}.

Visualization of a quasi-sphere takes long time.
We developed an algorithm to render this kind of fractals fast
\cite{bridges2018}.
It is called \textit{Iterated Inversion System (IIS)}.

In this paper, we briefly introduce sphairahedron and its rendering
technique.
Next, we show problems found by viewing masthematical problems.
We find mathematical problems and solve the problems.


We are interested in finding examples of quasi-fuchsian sub-group of
M\"ob$_{+}(S^3)$, the  group of orientation preserving M\"obius
transformation on $S^3 = R^3 \bigcup \{\infty\}$. For a subgroup $G$ of
Mob$_+(S^3)$, let the discontinuity set $\Omega(G)$ be
$$
\Omega(G) = \left\{ x \in S^3 | \text{the point x possesses a neighborhood} U(x)
\text{such that the intersection} U(x) \cap gU(x) \text{is empty for all but finite
elements} g\in G. \right\}.
$$
The complement $\Lambda(G) := S^3 \backslash \Omega(G)$ is called the
limit set of the group $G$. $G$$ is a Kleinian group if $\Omega(G)$ is
not empty. The limit set $\Lambda(G)$ consists either of 0 or 1 or 2 or
infinitely many points. Here we consider only the case that $\Lambda$
is infinite and two-dimensional.

\section{Preparation}

definition of a sphairahedra.
See also \cite{bridges2018}.

\section{Visualization}
We use ray tracing technique.
Ray marching
See also \cite{bridges2018}
3D print
\section{Problems}
\section{Summary \& Future Work}

章立て
Introduction 球面体群について.これまでの研究成果について阿原・荒木
(classification problem),蔭山の成果
Preparation 球面体の定義など 離散的,非離散的について 理想的かつ有理的
Visualization 可視化方法について
Problems (Propositions) 図を見ることで考察することのできる数学的問題を提起,解決
Summary \& Future work 解決したこと,未解決問題のまとめ

単連結な部分が0のときはどうする? 最初から抜いておく?

\cite{bridges2018}では,球面体の定義とその図像の豊かさについて論じた.


Kazushi Ahara prooved that the limit set of the sphairahedra
is a union of the spheres.

球面体の定義
二次元における円辺形によるフラクタル(クライン群)について

(阿原・荒木がやったこと)
Yoshiaki Araki and Kazushi Ahara found Sphairahedron in 2003.
It is extension of two dimensional tessellation of circles.
円辺形を拡張した図形である球面体を用いる.
cubeタイプのパラメータに関して計算を行ない,パラメータスペースを求めた.
その求め方の論文は出版されていない.
球面体そのものでなく,球面体を含む球の和集合に群を作用させることによっても
quasi-sphereを得ることができる.その証明は...
立方体をベースにした球面体群の分類問題を扱った.
今回はレンダリングされた図から得られた数学的問題を解決する

Kazushi Ahara prooved that the limit set of the sphairahedra
is a union of the spheres.

(蔭山さんがやったこと)
In 2016 Ryo Kageyama had supplementary examination in his master thesis.
He calculate parameter space of the cube type sphairahedron.
三次元でレンズ空間を考えた.
阿原・荒木と別にcubeタイプのパラメータスペースを求めた
日本語論文である

(中村がやったこと)
球面体群の可視化は長らく時間のかかる処理であった.
IISによってリアルタイムに可視化することができるようになった.
Visualize sphairahedron(semi-sphairahedron)-based fractals in real time.
The limit set is many spheres touching each other and,
and it becomes not Fuchsian group,
準球面体の可視化にも成功した.
semi-sphairahedronの極限集合は球が多数接した形状になっており,フックス群
でなくなる.
The limit set of the semi-sphairahedron
Kento Nakamura compute parameter spaces for other sphairahedra.
別の立体におけるパラメータ表示を計算し,新たな形状を見ることができた

可視化によって見えること
Ahara Arakiによって計算されたパラメータスペースの外側にも
離散的になるパラメータが混在してい可能性がある.
それは点か,曲線か,領域か,わかっていない
There may be parameters which is discrete.
We don't know it is point, curved line, or regions.

ビジュアライズされて図から読みとれること
固定点付近は収束が遅いこと
橋ができたときにPI/nになった,
球面体の各頂点の収束が特に遅い(タイリングするためにはたくさんの球面体が必要)

七角形以上のSphairahedronはパターンが多くなること,条件を満たす
パラメータが厳しくなることから,計算することが難しい
角度の条件から
五つ以上の辺が繋がる頂点があってはいけない


(まだできていないこと)
Find mathematical property of the fractals.
実験が必要?
地形タイプの場合どこまで高くなるか(低くなるか)
橋がかかった場合の離散条件(pi/n)
The limit set of the semi-sphairahedron is not known.
However, we visualize semi-sphairahedron and its tessellation.

図を観察して こういうことができるという主張
パラメータ空間の形状はフラクタル形状がない
パラメータスペースが連結

ゴーストサークルの概念が三次元にもある
ゴーストスフィア


\begin{thebibliography}{99}
\bibitem{AharaAraki}
        Kazushi Ahara and Yoshiaki Araki,
        \emph{Sphairahedral approach to parameterize visible three
        dimensional quasi-Fuchsian fractals},
        Proceedings of Computer Graphics International, 2003, pp. 226--229.
\bibitem{bridges2018}
        Kento Nakamura and Kazushi Ahara,
        \emph{Sphairahedra and Three-Dimensional Fractals}, 
        Proceedings of Bridges 2018: Mathematics, Music, Art, Architecture,
        Education, Culture, Tessellations Publishing,
        Phoenix, Arizona, 2017, pp. 171--178.
\end{thebibliography}
\end{document}
